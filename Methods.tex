\chapter{Methods}

    Three different methods for determining relative percentages of the dopant
     in the substrate are used. The methods are: 1. Take the ratio of the max 
     peak intensity, 2. Fit the peaks to a Gaussian curve, and find the integrated 
     area, 3. Perform a summation to find the numerical integral. These methods 
     are predicated on certain assumptions. The mixture of the probed depth is 
     uniform, which is a known simplification of how the diffused dopants actually 
     spread (an analysis to correct this has been tried, but better data is needed 
     for it to be effective). The intensity of the x-ray is also assumed to have 
     attenuated by 90\%, and all relevant data comes from that. The last assumption 
     is that the sample is put in its preferred orientation for the silicon structure 
     to be alined with the x-ray beam. The index of refraction is the same as bulk samples\cite{COLOMBI2007554}.
    
    \subsubsection*{Method 1: Intensity}

        With this method the program takes the maximum intensity of the amorphous hump, and the maximum intensity 
        of the most intense crystalline peak, and takes the ratio of them. 
        This ratio is then the relative percentages of the dopant and substrate\cite{pandey_structural_2021}.

    \subsubsection*{Method 2: Curve Fitting}

        This method takes the position and intensity values as the x and y inputs of scipy's curve\_fit function. 
        It then outputs the best fit for amplitude, mean, standard deviation, and a background value. 
        The function then gets integrated, and the ratio of the areas are taken.

    \subsubsection*{Method 3: Summation}

        This is an integral method as in method 2, but instead of fitting the data to a curve the data is 
        integrated using discrete methods. 
        The dx is the same for all points, so it is ignored with the foresight that it will be divided out. 
        The sum is of all the intensity values for the peak or hump. 
        A ratio is then taken from these sums.