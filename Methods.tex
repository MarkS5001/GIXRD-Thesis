\chapter{Methods}

    Three different methods for determining relative percentages of the dopant in the substrate are used. 
    The methods are named: 1. Intensity, 2. Curve Fitting, and 3. Summation. 
    Before discussing the intricacies of these methods, it is important to first know what shared simplifying assumptions on which they are predicated. 
    The first of these assumptions is that the mixture of the probed depth is uniform. 
    Uniformity of the mixture cannot be achieved in the sample due to the diffusion process starting at the top of the sample working its way down. %which is a known simplification of how the diffused dopants actually spread (an analysis to correct this has been tried, but better data is needed for it to be effective). 
    The next assumption is that the intensity of the x-ray has been attenuated by 90\%. 
    The thirdly, that the sample is put in its preferred orientation that allows for the silicon structure to be alined with the x-ray beam. 
    The last assumption is the index of refraction is the same as bulk samples\cite{COLOMBI2007554} as those values will be utilized.

    \subsubsection*{Method 1: Intensity}

        This method takes the highest count of the amorphous hump, and the highest count of the crystalline peaks as input parameters, to take the ratio of them.
        The following formula is used for the calculation:
        $$m=\frac{I_a}{I_a+I_c}$$
        $I_a$ is the highest count of the amorphous hump, $I_c$ is the highest count of the crystalline peaks, and $m$ is the mixture percent.

        % With this method the program takes the maximum intensity of the amorphous hump, and the maximum intensity 
        % of the most intense crystalline peak, and takes the ratio of them. 
        % This ratio is then the relative percentages of the dopant and substrate\cite{pandey_structural_2021}.

    \subsubsection*{Method 2: Curve Fitting}

        The amorphous hump, and the crystalline peaks can be approximated by Gaussian curves. 
        As such, it is possible to fit a curve to the hump, and the desired peak. 
        With this fit, analytic integration is performed. 
        The values from the integrations are then taken as a ratio as in the intensity method. 
        This method also gives a background baseline that can be subtracted out.

        % This method takes the position and intensity values as the x and y inputs of scipy's curve\_fit function. 
        % It then outputs the best fit for amplitude, mean, standard deviation, and a background value. 
        % The function then gets integrated, and the ratio of the areas are taken.

    \subsubsection*{Method 3: Summation}

        This method is similar to the curve fitting method, but instead of fitting the hump, and peak to a Gaussian, a numerical integration is performed. 
        This numerical integration sums up the heights in the region of the relevant area to take their ratio. 
        As stated, only the heights are summed; This is due to the spacing between all points being identical. 
        Below is the proof: 
        $$m=\frac{\sum_{\theta_i}^{\theta_f}y_adx_a}{\sum_{\theta_i}^{\theta_f}y_adx_a+\sum_{\theta_i}^{\theta_f}y_cdx_c}$$
        $$dx_a=dx_c=Constant$$
        $$m=\frac{dx_a\sum_{\theta_i}^{\theta_f}y_a}{dx_a\sum_{\theta_i}^{\theta_f}y_a+dx_c\sum_{\theta_i}^{\theta_f}y_c}$$
        $$m=\frac{dx_a\sum_{\theta_i}^{\theta_f}y_a}{dx_a\left(\sum_{\theta_i}^{\theta_f}y_a+\sum_{\theta_i}^{\theta_f}y_c\right)}$$
        $$m=\frac{\sum_{\theta_i}^{\theta_f}y_a}{\sum_{\theta_i}^{\theta_f}y_a+\sum_{\theta_i}^{\theta_f}y_c}$$

        % This is an integral method as in method 2, but instead of fitting the data to a curve the data is 
        % integrated using discrete methods. 
        % The dx is the same for all points, so it is ignored with the foresight that it will be divided out. 
        % The sum is of all the intensity values for the peak or hump. 
        % A ratio is then taken from these sums.