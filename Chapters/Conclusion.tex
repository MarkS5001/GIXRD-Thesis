\chapter{Conclusion}

    Using GIXRD to determine relative composition of samples led to three analysis methods. 
    Two of the methods utilized integration, and their results are more closely alined. 
    One method used peak intensity ratios, and provides results that noticeably differ. 
    All three methods provide reasonable agreement as to the depth that each omega probed. 
    Using GIXRD to see relative mixture amount dependent on depth has been shown to be useful, 
    keeping in mind the limitations given by the assumptions.

% \section{Future Work}

%     Performing this analysis on a known sample would be able to provide greater validity to the results of this analysis.
%     Creating an analysis to correct for the real diffusion process by not assuming uniform layers.
%     This would produce the greatest impact on the real world application of these results.
    
    % Reformat the below to go into future work
    %The first of these assumptions is that the mixture of the probed depth is uniform, which is a known simplification of how the diffused dopants actually spread (an analysis to correct this has been tried, but better data is needed for it to be effective)