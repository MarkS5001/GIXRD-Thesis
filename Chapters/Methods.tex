\chapter{Methods}

    % Methods used to get the data
    % GIXRD holds the incident angle fixed
    % while moving the detector. Subsequent
    % scans are done with different incident
    % angles to probe different depths. Then by
    % looking at the different peak heights, and
    % breadths, one can make conclusions
    % about the relative percentages of the
    % material above it.

    The process by which the XRD is used in this work to probe increasingly deeper depths is called grazing incidence x-ray diffraction (GIXRD; also called glancing incidence x-ray diffraction).
    GIXRD holds an incident angle relative to the surface of the sample (called omega ($\omega$)), while the detector on the diffracted beam side sweeps over a range of angles (figure \ref{fig:GIXRD diagram}).
    The detector is used in 0D mode in conjunction with a parallel plate collimator; This means it simply counts all the x-rays that interact with it.
    The counts get added to bins corresponding to the current detector angle according to this equation: $2(\omega+\theta)$, with $\theta$ being the current detector angle.
    After the detector has swept through all programed angles, a new scan is started with a different $\omega$ corresponding to a different depth.
    % The XRD process used to incrementally probe deeper depths is called grazing incidence x-ray diffraction (GIXRD, also called glancing incidence x-ray diffraction).
    % Using this method the XRD will hold the incident beam at a fixed angle (called omega ($\omega$)), while the diffracted beam side moves from a starting angle to an end angle.
    % The detector is used in a 0D mode that simply counts all the x-rays that interact with it.
    % The angle the counts correspond to is the angle the incident beam is added to the current diffracted beam angle all multiplied by 2: $2(\omega+\theta)$.
    % After the scan finishes a new scan is started with a different incident angle to probe a different depth.

    The different incident angles probe different depths due to the different path lengths they provide the x-rays as they interact with the samples. 
    This is due to the exponential attenuation of intensity as the beam travels given by this equation: $I(y)=I_0e^{-\alpha y}$ with $\alpha=2\omega n_I/c$\cite{https://doi.org/10.1002/jps.22202,hecht}. 
    With the complex index of refraction ($n_I$) strongly dependent on wavelength. 
    The x-ray source is copper, producing $K_\alpha$, and $K_\beta$ in most abundance. 
    Since one wavelength provides the most uniformity on the diffractogram, the incident beam optics include a filter to limit the unnecessary wavelength. 
    $K_\alpha$ is what is most common, so the $K_\beta$ gets mostly filtered out by a nickel filter. 
    This results in a x-ray beam that has a wavelength of 1.54\AA (uniformity in the wavelength is also necessary for Bragg's law).
    It is these subtle differences that allow for different ratios of dopant to substrate.

    % Picture of GIXRD schematic
    \begin{figure}[H]
        \centering
        \includesvg[width=1\columnwidth]{Pictures/2theta scan diagram.svg}
        \caption{Diagram of XRD utilizing GIXRD}
        \label{fig:GIXRD diagram}
    \end{figure}

    % Methods used to get percentages
    Three different methods for determining relative percentages of the dopant in the substrate are used.
    The methods are named: 1. Intensity, 2. Curve Fitting, and 3. Summation.
    Before discussing the intricacies of these methods, it is important to first know what shared simplifying assumptions on which they are predicated.
    The first of these assumptions is that the mixture of the probed depth is uniform.
    Uniformity of the mixture cannot be achieved in the sample due to the diffusion process starting at the top of the sample working its way down.%which is a known simplification of how the diffused dopants actually spread (an analysis to correct this has been tried, but better data is needed for it to be effective).
    However, this assumption is needed as the basis for future work to use as a starting point.
    The next assumption is that the intensity of the x-ray has been attenuated by 90\%.
    The thirdly, that the sample is put in its preferred orientation that allows for the silicon structure to be alined with the x-ray beam.
    The last assumption is that the index of refraction is the same as bulk samples\cite{COLOMBI2007554} as those values will be utilized.

    \subsubsection*{Method 1: Intensity}

        This method takes the highest count of the amorphous hump, and the highest count of the crystalline peaks as input parameters, to take the ratio of them.
        This ratio is then the relative percentages of the dopant and substrate\cite{pandey_structural_2021}.
        The following formula is used for the calculation:
        $$m=\frac{I_a}{I_a+I_c}$$
        $I_a$ is the highest count of the amorphous hump, $I_c$ is the highest count of the crystalline peaks, and $m$ is the mixture percent.

        % With this method the program takes the maximum intensity of the amorphous hump, and the maximum intensity 
        % of the most intense crystalline peak, and takes the ratio of them.
        % This ratio is then the relative percentages of the dopant and substrate\cite{pandey_structural_2021}.

    \subsubsection*{Method 2: Curve Fitting}

        The amorphous hump, and the crystalline peaks can be approximated by Gaussian curves.
        As such, it is possible to fit a curve to the hump, and the desired peak.
        With this fit, analytic integration is performed.
        The values from the integrations are then taken as a ratio as in the intensity method.
        This method also gives a value to the background noise that can then be subtracted out.

        % This method takes the position and intensity values as the x and y inputs of scipy's curve\_fit function.
        % It then outputs the best fit for amplitude, mean, standard deviation, and a background value.
        % The function then gets integrated, and the ratio of the areas are taken.

    \subsubsection*{Method 3: Summation}

        This method is similar to the curve fitting method, but instead of fitting the hump, and peak to a Gaussian, a numerical integration is performed.
        This numerical integration sums up the heights in the region of the relevant area to take their ratio.
        As stated, only the heights are summed; This is due to the spacing between all points being identical.
        Below is the proof: 
        $$m=\frac{\sum_{\theta_i}^{\theta_f}y_adx_a}{\sum_{\theta_i}^{\theta_f}y_adx_a+\sum_{\theta_i}^{\theta_f}y_cdx_c}$$
        $$dx_a=dx_c=Constant$$
        $$m=\frac{dx_a\sum_{\theta_i}^{\theta_f}y_a}{dx_a\sum_{\theta_i}^{\theta_f}y_a+dx_c\sum_{\theta_i}^{\theta_f}y_c}$$
        $$m=\frac{dx_a\sum_{\theta_i}^{\theta_f}y_a}{dx_a\left(\sum_{\theta_i}^{\theta_f}y_a+\sum_{\theta_i}^{\theta_f}y_c\right)}$$
        $$m=\frac{\sum_{\theta_i}^{\theta_f}y_a}{\sum_{\theta_i}^{\theta_f}y_a+\sum_{\theta_i}^{\theta_f}y_c}$$
        $y_a$, and $y_c$ is the count of the current $theta$ position for the amorphous and crystalline parts respectively.
        $dx_a$, and $dx_c$ is the spacing between the $theta$ positions for the amorphous and crystalline parts respectively.
        $m$ is the mixture amount, and $theta_f$, and $theta_i$ are the starting and ending $theta$ positions for the amorphous hump or crystalline peak respectively.

        % This is an integral method as in method 2, but instead of fitting the data to a curve the data is 
        % integrated using discrete methods.
        % The dx is the same for all points, so it is ignored with the foresight that it will be divided out.
        % The sum is of all the intensity values for the peak or hump.
        % A ratio is then taken from these sums.

        A similar proof can also be performed to show the counts do not need to be normalized.