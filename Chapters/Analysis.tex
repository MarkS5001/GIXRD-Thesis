\chapter{Analysis} % Potentially talk about uncertainty here or in the methods chapter

    \begin{figure}[H]
        \centering
        \includesvg[width=1\linewidth]{Pictures/Percentages with Uncertainty.svg}
        \caption{Percentage of amorphous vs omega for the three methods with fractional uncertainty}
        \label{fig:line graph amorphous with errorbars}
    \end{figure}
        
    \begin{figure}[H]
        \centering
        \includegraphics[width=.75\linewidth]{Pictures/Amorphous Percent with Uncertainty.pdf}
        \caption{Table for percentage of amorphous vs omega for the three methods with fractional uncertainty}
        \label{fig:table amorphous with errorbars}
    \end{figure}

    From figure \ref{fig:line graph amorphous with errorbars} we can see that the intensity method is not in agreement with the curve fit or summation methods.
    The curve fit, and summation methods are expected to be in agreement with each other since the methods are both integral methods.
    From the table in figure \ref{fig:table amorphous with errorbars} we can see, with more precision, that the uncertainty is lower as well for the summation method over the intensity method.
    The $0.005^\circ$ shows that that depth is composed of a value greater than unity for the sample; This is physically impossible.
    The greater than unity error comes about from having no crystalline peak from the silicon, so the amorphous hump from the phosphorous is the only thing present on the diffractogram (this is the diffractogram in figure \ref{fig:pure amorphous XRD scan}).

    % Figure \ref{fig:line graph amorphous with errorbars} provides a concise overview of how the dopant and substrate mixture change with omega.
    % All three methods show the amorphous percent decreases with depth, which is to be expected.
    % They also show that the 0.005 omega is greater than 1 for the amorphous.
    % This is due to the crystalline peak being nonexistent, but the program is still trying to see a peak.
    % The curve fitting and summation method provide excellent agreement with each other, while the ratio method alludes to an exponential.

    \begin{figure}[H]
        \centering
        \includegraphics[width=1\linewidth]{Pictures/Depth vs Omega.png}
        \caption{Depth vs omega for the three methods}
        \label{fig:depth vs omega}
    \end{figure}

    In Figure \ref{fig:depth vs omega}, we can see the depth that correlates to the three methods.
    The three methods have close agreement at lower depths before splitting between the intensity method and the curve fit and summation methods.
    The curve fit and summation method in particular are right on top of each other.
    % Figure \ref{fig:depth vs omega} shows the depth that correlates with the three methods.
    % The methods have a close agreement, with the curve fit and summation method being right on top of each other.
    % They all show a mostly linear relationship, which is to be expected since the indices of refraction are very similar for the dopant and substrate.
    % Nevertheless, it is interesting that the methods more closely agree at lower and high omegas, but not the middle omegas.