\chapter{Background}

    Solids can be divided into two broad types: crystalline, and amorphous. 
    Crystalline solids have repeating structure that is predictable, and is consistent throughout the entire solid. 
    Amorphous solids have no repeating structures, and have no preferred orientation.
    
    X-Ray diffraction (XRD) utilizes x-rays to non-destructively probe the inner atomic layers of materials. 
    In this work it is used to probe different depths of doped silicon to see dopant levels as a function of depth. 
    As the x-ray interacts with the sample it will reflect, refract, and diffract. 
    Diffraction is the process of note, and the only one that will be investigated for this work. 

    When XRD is used on crystalline solids a diffraction pattern will emerge. 
    This diffraction pattern arises from Bragg's law:
    $$2d\sin\theta=n\lambda$$
    $\lambda$ is the wavelength of the light.% (Cu $k_\alpha$ of wavelength 1.54\AA is used in this work).
    $d$ is the slit spacing, which are the interatomic spacings.
    These interatomic spacings are also the spacings between different planes of atoms as well, instead of only nearest neighbors.
    These planes of atoms are then relegated to act as a diffraction grating for the incident light, causing constructive and deconstructive interference with all other planes.
    % The spacing between the atoms acts as the slit size that allows for the diffraction interference pattern, in accordance with Bragg's Law ($2d\sin\theta=n\lambda$). 
    % There also exist different spacings between atoms that are not nearest neighbors. 
    % Each of these different spacings are relegated into planes that act like a diffraction grating, existing for every possible repeating spacing within the atom. 
    All of these in superposition produce a graph that shows sharp peaks correlating directly to the atomic plane's spacing (shown in figure \ref{fig:pure crystalline XRD scan}).
    
% Show picture of crystalline XRD pattern
\begin{figure}[H]
    \centering
    \includegraphics[width=\linewidth]{Pictures/crystalline XRD scan.png}
    \caption{Diffractogram of pure crystalline solid}
    \label{fig:pure crystalline XRD scan}
\end{figure}

    Amorphous solids under the x-ray beam behave in a similar manner. 
    The difference being because every atom is arranged randomly, there are an infinite number of planes all with their own spacing. 
    As shown in figure \ref{fig:pure amorphous XRD scan}, this correlates to a moderately low angle hump (called an amorphous hump) with all spacings that would correlate to higher angled peaks being nonexistent because of deconstructive interference\cite{PARKS2007277,achilles_amorphous_2018,ma18092093}.

% Show picture of amorphous XRD pattern
\begin{figure}[H]
    \centering
    \includegraphics[width=1\linewidth]{Pictures/amorphous XRD scan.png}
    \caption{Diffractogram of pure amorphous solid}
    \label{fig:pure amorphous XRD scan}
\end{figure}

    The materials analyzed in this work are silicon doped with phosphorous, silicon doped with boron, and pure silicon. 
    The first two materials will be best thought of as a gradient from pure phosphorous or boron to pure silicon from top to bottom. 
    Silicon is a crystalline material, while phosphorous and boron are amorphous. 
    This means that the diffractogram will be a superposition of an amorphous diffractogram on a crystalline diffractogram. 
    The silicon sample was only used to verify that the amorphous hump was present on the diffractograms.